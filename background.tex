\chapter{Background}

\section{Matroid theory}
\label{sec:matroid-theory}
If a mathematical structure can be defined or axiomatized in multiple different, but not obviously equivalent, ways, the different definitions or axiomatizations of that structure make up a cryptomorphism. The many obtusely equivalent definitions of a matroid are a classic example of cryptomorphism, and belie the fact that the matroid is a generalization of concepts in many, seemingly disparate areas of mathematics.

First introduced by Hassler Whitney in 1935~\cite{whitney-1935}, in a seminal paper where he described two axioms for independence in the columns of a matrix, and defined any system obeying these axioms to be a ``matroid''. Whitney's key insight was that this abstraction of~~``independence'' is applicable to both matrices and graphs. As a result of this, the terms used in matroid theory are borrowed from analogous concepts in both graph theory and linear algebra. Matroids have also received attention from researchers in fair allocation, as their properties make them useful for modeling user preferences; for instance, matroid rank functions are a natural way of formally describing course allocation for students~\cite{benabbou-2021}. 

\subsection*{Independent sets}
Perhaps the most common way to define a matroid is in terms of its \textit{independent sets}. An independence system is a pair $(E, \mathcal{I})$, where $E$ is the ground set of elements, $E \not= \emptyset$, and $\mathcal{I}$ is the set of independent sets, $\mathcal{I} \subseteq 2^E$. A matroid is an independence system with the following properties:

\begin{enumerate}
  \item The empty set is independent: $\emptyset \in \mathcal{I}$.
  \item The hereditary property: if $A \subseteq B$ and $B \in \mathcal{I}$, then $A \in \mathcal{I}$.
  \item The augmentation property: If $A, B \in \mathcal{I}$ and $|A| > |B|,$ then there exists $e \in A$ such that $B \cup \{e\} \in S$.
\end{enumerate}

In practice, the ground set $E$ represents the universe of elements in play, and the independent sets of typically represent the legal combinations of these items. In the context of fair allocation, the independent sets represent the legal (in the case of matroid constraints) or desired (in the case of matroid utilities) bundles of items.

\subsection*{Rank}
Given a matroid $\mathfrak{M} = (E, \mathcal{I})$, the \textit{matroid rank function} (MRF) is a function $\fn{rank} : 2^E \to \mathbb{Z}^+$ that gives the \textit{rank} of a set $ A \subseteq E$, which is defined to be the size of the largest independent subset of $A$. Formally, $$\fn{rank}(A) = \max\{|X| : X \subseteq A \text{ and } X \in \mathcal{I}\}.$$ Matroid rank functions are \textit{binary submodular}. Binary because they have binary marginals, that is, $\fn{rank}(A \cup \{ e \}) - \fn{rank}(A) \in \{0,1\}$, for all $A \subseteq 2^E$ and $e \in E$. Submodularity refers to rank functions' natural diminishing returns property, namely that for any two sets $X, Y \subseteq E$, we have $$\fn{rank}(X \cup Y) + \fn{rank}(X \cap Y) \leq \fn{rank}(X) + \fn{rank}(Y).$$ This diminishing returns property makes the rank function useful for modeling user preferences, and is a reason why matroids show up so often in economics and game theory (???).

\subsection*{Closed sets}

We also need to establish the concept of the \textit{closed sets} of a matroid. A closed set is a set whose cardinality is maximal for its rank. Equivalently to the definition given above, we can define a matroid as $\mathfrak{M} = (E, \mathcal{F})$, where $\mathcal{F}$ is the set of closed sets of $\mathfrak{M}$, satisfying the following properties:

\begin{enumerate}
  \item The set of all elements is closed: $E \in \mathcal{F}$
  \item The intersection of two closed sets is a closed set: If $A,B \in \mathcal{F},$ then $A \cap B \in \mathcal{F}$
  \item If $A \in \mathcal{F}$ and $a,b \in E \setminus A,$ then $b$ is a member of all sets in $\mathcal{F}$ containing $A \cup \{a\}$ if and only if $a$ is a member of all sets in $\mathcal{F}$ containing $A \cup \{b\}$
\end{enumerate}

\section{Examples of matroids}
\skelpar

\subsection*{The free matroid}
\skelpar

\subsection*{The uniform matroid}
\skelpar

\subsection*{The vector matroid}
\skelpars[2]

\subsection*{The graphic matroid}
\skelpars[3]