\chapter{The development of \texorpdfstring{\jlinl{random\_kmc}}{random\_kmc}}
\label{apx:rkmc_dev}
This final appendix is where I have stowed away the lengthier bits of code referred to in Section~\ref{sec:improving-performance}, detailing the development of a somewhat performant implementation of \pr{Random-Knuth-Matroid} (Algorithm~\ref{alg:rkmc}).

\begin{figure}
  \begin{jllisting}
function generate_covers_v1(Fr, E)
  Set([A ∪ a for A ∈ Fr for a ∈ setdiff(E, A)])
end

function superpose_v1!(F, F_old)
  for A ∈ F, B ∈ F
    should_merge = true
    for C ∈ F_old if A ∩ B ⊆ C
      should_merge = false
    end end

    if should_merge
      setdiff!(F, [A, B])
      push!(F, A ∪ B)
    end
  end

  return F
end
  \end{jllisting}
  \caption{Initial implementation \jlinl{generate_covers} and \jlinl{superpose}.}
  \label{code:rkmc_v1_a}
\end{figure}

\begin{figure}
  \begin{jllisting}
function random_kmc_v1(n, p, T)
  E = Set([i for i in range(0,n-1)])
  
  # Step 1: Initialize.
  r = 1
  F = [family([])]
  pr = 0
  
  while true
    # Step 2: Generate covers.
    push!(F, generate_covers_v1(F[r], E))
    
    # Step 4: Superpose.
    superpose_v1!(F[r+1], F[r])
    
    # Step 5: Test for completion.
    if E ∈ F[r+1]
      return KnuthMatroid{Set{Integer}}(n, F, [], Set(), Dict())
    end
    
    if r <= length(p)
      pr = p[r]
    end
    
    while pr > 0
      # Random closed set in F_{r+1} and element in E \ A.
      A = rand(F[r+1])
      a = rand(setdiff(E, A))
      
      # Replace A with A ∪ {a}.
      F[r+1] = setdiff(F[r+1], A) ∪ Set([A ∪ a])
      
      # Superpose again to account for coarsening step.
      superpose_v1!(F[r+1], F[r])
      
      
      # Step 5: Test for completion.
      if E ∈ F[r+1]
        return (E, F)
      end
      
      pr -= 1
    end
    
    
    r += 1
  end
end
  \end{jllisting}
  \caption{\jlinl{random_kmc_v1}}
  \label{code:rkmc_v1_b}
\end{figure}

\begin{figure}
  \begin{jllisting}
function bitwise_superpose!(F, F_prev)
  i = 0
  As = copy(F)
  while length(As) !== 0
    A = pop!(As)

    for B in setdiff(F, A)
      i += 1
      if should_merge(A, B, F_prev)
        push!(As, A | B)
        setdiff!(F, [A, B])
        push!(F, A | B)
        break
      end
    end
  end

  return F
end

function generate_covers_v2(F_r, n)
  Set([A | 1 << i for A ∈ F_r for i in 0:n-1 if A & 1 << i === 0])
end
  \end{jllisting}
  \caption{The bitset-implementations of \pr{Generate-Covers} and \pr{Superpose!}, first used in \jlinl{random_kmc_v2}.}
  \label{code:rkmc_v2}
\end{figure}

\begin{figure}
  \begin{jllisting}
function sorted_bitwise_superpose!(F, F_prev)
  As = sort!(collect(F), by = s -> length(bits_to_set(s)))
  while length(As) !== 0
    A = popfirst!(As)

    for B in setdiff(F, A)
      if should_merge(A, B, F_prev)
        insert!(As, 1, A | B)
        setdiff!(F, [A, B])
        push!(F, A | B)
        break
      end
    end
  end

  return F
end
  \end{jllisting}
  \caption{This implementation of \pr{Superpose!} sorts the sets by length.}
  \label{code:rkmc_v3}
\end{figure}


\begin{figure}
  \begin{jllisting}
function random_kmc_v4(n, p, T=UInt16)::ClosedSetsMatroid{T}
  r = 1
  pr = 0
  F = [Set(0)]
  E = 2^n - 1 # The set of all elements in E.

  while true
    to_insert = generate_covers_v2(F[r], n)

    # Apply coarsening to covers.
    if r <= length(p) && E ∉ to_insert # No need to coarsen if E is added.
      pr = p[r]
      while pr > 0
        A = rand(to_insert)
        a = random_element(E - A)
        to_insert = setdiff(to_insert, A) ∪ [A | a]
        pr -= 1
      end
    end

    # Superpose.
    push!(F, Set()) # Add F[r+1].
    while length(to_insert) > 0
      A = pop!(to_insert)
      push!(F[r+1], A)

      for B in setdiff(F[r+1], A)
        if should_merge(A, B, F[r])
          push!(to_insert, A | B)
          setdiff!(F[r+1], [A, B])
          push!(F[r+1], A | B)
        end
      end
    end

    if E ∈ F[r+1]
      return ClosedSetsMatroid{T}(n, r, F, Dict(), T)
    end

    r += 1
  end
end
  \end{jllisting}
  \caption{\jlinl{random_kmc_v4} inserts the sets one at a time, superposing on the fly.}
  \label{code:rkmc_v4}
\end{figure}

\begin{figure}
  \begin{jllisting}
function random_knuth_matroid(n, p, T=UInt16)::ClosedSetsMatroid{T}
  r = 1
  F::Vector{Set{T}} = [Set(T(0))]
  E::T = BigInt(2)^n-1
  rank = Dict{T, Integer}(0=>0)

  while E ∉ F[r]
    # Initialize F[r+1].
    push!(F, Set())

    # Setup add_set.
    add_callback = x -> rank[x] = r
    add_function = x -> add_set!(x, F, r, rank, add_callback)

    generate_covers!(F, r, E, add_function)

    # Perform coarsening.
    if r <= length(p) coarsen!(F, r, E, p[r], add_function) end

    r += 1
  end

  return ClosedSetsMatroid{T}(n, r-1, F, rank, T)
end
  \end{jllisting}
  \caption{The final implementation of \pr{Random-Knuth-Matroid}.}
  \label{code:random_knuth_matroid}
\end{figure}