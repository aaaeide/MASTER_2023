\chapter{Concluding remarks}
\label{chap:conclusions}
Initially, all I knew about this thesis was that it was going to have something to do with fair allocation. Looking around for recent allocation algorithms that whose study might form part of a thesis, Viswanathan and Zick's Yankee Swap algorithm caught my attention. By restricting the valuations to the class of matroid rank functions, a seemingly simple algorithm could deliver extraordinarily well on a range of fairness objectives intractable in the general, additive case. My interest piqued, I wanted to understand how it worked, and set about implementing the algorithm using Hummel and Hetland's Allocations.jl library. Almost immediately I was flummoxed by how to represent the matroid rank valuations. I had assumed that there would exist some library to facilitate working programmatically with matroids, similar to how Graphs.jl enables working with graphs without needing to re-implement the wheel graph. At the time I was unable to find any such library, and so the research question for this thesis came to be: how might one design and implement a Julia library to support the implementation of and experimentation with matroidal fair allocation algorithms? 

The primary goal of this thesis was to build Matroids.jl as an answer to that question---a practical tool to complement the theoretical toolkit provided by matroid theory. An important sub-goal was to figure out how to make Matroids.jl performant; as we have seen, matroids permit many powerful polynomial-time operations, such as the matroid partition algorithm, that papers on matroidal fair allocation algorithms use in their analyses to show that allocations can be found efficiently. This obscures many implementation-level optimization decisions that can drastically improve the practical runtime of the implemented algorithms. One example of such an optimization is to use the rank function as sparingly as possible, in favor of cheaper independence or cardinality checks, as I discuss when giving some algorithm implementations in Chapter~\ref{chap:yankee-swap}.

Late in the project, I realized that there does in fact exist matroid libraries in Julia, in all likelihood vastly more performant and feature-rich than Matroids.jl would ever be\footnote{See for instance \href{https://docs.oscar-system.org/stable/Combinatorics/matroids/}{https://docs.oscar-system.org/stable/Combinatorics/matroids/}.}. The primary target demographic for Matroids.jl had always been fair allocation researchers, but upon witnessing the capabilities of my more advanced competitors, the ``secret'' secondary target demographic came to the fore, namely students, computer programmers and non-mathematicians such as myself. All along, I realized, I had been building the library for myself, the library that I had needed when I wanted to figure out how Yankee Swap worked, which was a simple-to-use matroid library that only concerned itself with the most basic aspects of matroids as they related to fair  allocation.

While matroid theory might seem an extremely abstract and niche subfield of mathematics, it has found applicability in the field of fair allocation, which in the end deals with problems of a highly practical and everyday nature. The aim of fair allocation, to deliver provably fair mechanisms for the distribution of resources, is a noble goal, and if a problem permits a matroidal representation, it can utilize algorithms that deliver very well indeed. If Matroids.jl is able to serve as a soft introduction to matroid theory for a computer programmer interested in understanding fair allocation algorithms, and if that computer programmer goes on to build a real-world solution for fair allocation, then Matroids.jl has achieved its goals as far as I am concerned.

\section{Limitations}
What types of user preferences are we unable to represent using MRFs? Ref. Halpern's introduction; MRFs can model substitutes, but not complementaries (left shoe worthless without right shoe).

Little support exists in Julia for working with Multigraphs at the moment.

\skelpar
\section{Future work}
\begin{enumerate}
  \item Exchange graphs, shortest path, augmentation. Stop recalculating everything after a transfer. Keep track of shortest paths, most will not change.
  \item Parallelization?
  \item Eliciting actual user preferences as matroids
\end{enumerate}
\skelpar