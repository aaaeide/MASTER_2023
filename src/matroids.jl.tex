\chapter{The Matroids.jl API}
A goal for this project is to introduce Matroids.jl as a useful library for experimenting with fair allocation that require matroids. In the previous chapter, we explored how such a library might generate and represent matroids, but this is not especially worthwhile until we also have in place an API layer to allow fair allocation algorithms to interface with our matroids in a practical and efficient manner.

\section{The independence oracle}
Whenever matroids show up in the context of fair allocation, the existence is assumed of an \textit{independence oracle}, which can in polynomial time (with respect to the number of elements) decide whether a set is independent. 


\section{The matroid union algorithm}
\skelpar


\section{Supporting universe sizes of \texorpdfstring{$n > 128$}{n > 128}}
The larger the ground set, the closer we are to an instance of The cake-cutting problem. Typical fair allocation problems with indivisible items deal with less than 100 items. \skelline{Referer til Spliddit og vanlige størrelser på fordelingsproblemer}

In other words, the Integer cap of 128 bits is a reasonable upper limit on universe size for fair allocation problems. However, one could look into using packages that add larger fixed-width integer types\footnote{See for instance \href{https://github.com/rfourquet/BitIntegers.jl}{\mono{BitIntegers.jl}}}. \mono{Matroids.jl} supports arbitrary integer types. \skelpars[1]{Beskriv åssen man kan oppgi valgfri Integer-type}