\chapter{The Matroids.jl API}
\label{chap:matroids.jl}
Matroids.jl exists to enable the empirical study of matroidal fair allocation. In this chapter, I consider what fair allocation-specific methods the Matroids.jl API should expose to achieve this goal, and how they might be implemented. While doing so, I keep track of which properties are required from the matroids being used. Chapter~\ref{chap:generating_matroids} describes how Matroids.jl generates a number of different matroid types, and how the getter functions for the properties we need are implemented.

The implementation will draw inspiration from, and be designed to integrate with, Hummel and Hetland's well-organized Allocations.jl library~\cite{Hetland_Allocations_jl_2022}, which provides a range of algorithms for fair allocation of indivisible items. Allocations.jl currently supports additive and submodular valuations and a number of constraint types, including conflict constraints and cardinality constraints (which, incidentally, are partition matroid constraints~\cite{barman2020fair}). Our goal is to extend Allocations.jl with support for matroid-rank valuations and matroidal constraints. As such, Matroids.jl should be structured in such a manner as to be familiar to those acquainted with Allocations.jl.

This chapter is structured as follows:
\begin{enumerate}
    \item I first extend Allocation.jl's existing functionality to support representing and evaluating fair allocation instances with matroid-rank valuations and matroidal constraints
    \item I then present the requirements of the algorithms bla bla bla
    \item I implement some stuff (matroid partition??)
\end{enumerate}

In the next chapter, I describe an implementation, using Matroids.jl, of the Yankee Swap algorithm for fair allocation with matroidal valuations~\cite{viswanathan2023yankee} and present some experimental results.

\section{Fairness under matroid-rank valuations}
If Matroids.jl is to be of use in the empirical study of matroidal fair allocation algorithms, we need to be able to evaluate the fairness of an allocation. In this chapter, I show how Matroids.jl implements the fairness criteria given in Chapter~\ref{chap:prelims}.

A matroid-rank-valued allocation problem instance is represented as a struct containing the matroids for each of the agents, along with the number of goods. Agent $i$'s value for the set of goods $S$, is the rank of $S$ in agent $i$'s matroid.
\begin{jllisting}
"""
    struct MatroidRank <: Profile

A matroid rank valuation profile, representing how each agent values all possible bundles. The profile is constructed from `n` matroids, one for each agents, each matroid over the set of goods [m]. 
"""
struct MatroidRank <: Profile
    matroids::Vector{Matroid}
    m::Int
end

value(V::MatroidRank, i, S) = rank(V.matroids[i], S)
value(V::MatroidRank, i, g::Int) = value(V, i, Set(g))
\end{jllisting}

\subsection*{Envy-freeness}
Checking if an allocation is EF is the same for matroid-rank valuations as for additive valuations -- simply compare each agent's own bundle value with that agent's subjective valuation of each other agent's bundle. This is already implemented in Allocations.jl. In this section, I give the functions \jlinl{value_1}, \jlinl{value_x} and \jlinl{value_x0}, which are used for computing EF1, EFX$_+$ and EFX$_0$, respectively. These functions take in a valuation profile, an agent $i$ and a bundle $S$, and return the agent $i$'s value for $S$, up to some item. 

To check efficiently if an allocation is EF1, we make use of the fact that a matroid rank function has binary marginals; in other words, the highest valued good in a bundle will always have value 1, unless the bundle value is 0. This gives us a simple way of checking for EF1. Similarly, since the least positively-valued good also has value 1, EFX$_+$ is the same as EF1. 
\begin{jllisting}
value_1(V::MatroidRank, i, S) = max(value(V, i, S)-1, 0)
value_x(V::MatroidRank, i, S) = value_1(V, i, S)
\end{jllisting}

The value of the least valued good overall (including 0-values) depend on whether the bundle is independent. An independent set contains by definition no redundant elements, so if the bundle is independent, the least-valued good has value 1. If the bundle is dependent, it contains at least one 0-valued good, or, equivalently, a good whose removal does not affect the bundle value. This gives us EFX$_0$.
\begin{jllisting}
value_x0(V::MatroidRank, i, A) =
    is_indep(V.matroids[i], A) ? value_1(V, i, A) : value(V, i, A)
\end{jllisting}


\subsection*{Proportionality}
To check whether an allocation $A$ is PROP or some relaxation thereof, we compare $v_i(A_i)$ against some threshold for every agent $i$. PROP$_i$ is simply the rank of the $\mathfrak{M}_i$, as this is the maximum value achievable for agent $i$, divided by the number of agents in the problem instance.
\begin{jllisting}
prop(V::MatroidRank, i, _) = rank(V.matroids[i])/na(V)
\end{jllisting}

To check for PROP1, we need to figure out if there exists some $g\in\mathcal{M}$ such that $v_i(A_i+g)\geq \frac{1}{n}v_i(\mathcal{M})$. We have two possible cases, the first being that $v_i(A_i) = v_i(\mathcal{M})$. This is trivially PROP1. In the other case, $v_i(A_i) < v_i(\mathcal{M})$. We know, due to the hereditary property (as given in Section~\ref{sec:indep_sets}) that there exists $g\in\mathcal{M}\setminus A_i$ such that $\Delta_i(A_i, g) = 1$ (this is equivalent to saying that $A_i$ does not a contain a basis of $\mathfrak{M}_i$). To figure out if $A$ is PROP1, then, we need to check whether $v_i(A_i) + 1 \geq \frac{1}{n}v_i(\mathcal{M}) \iff v_i(A_i) \geq \frac{1}{n}v_i(\mathcal{M}) - 1$.
\begin{jllisting}
prop_1(V::MatroidRank, i, A) = value(V, i, A) >= prop(V, i, A) - 1
\end{jllisting}

When checking for PROPX$_0$, we want the $g\in E\setminus A_i$ whose addition would increase the value of $A_i$ the least. If $A_i$ is a closed set, then any additional good will increase the rank by 1, otherwise there is an element