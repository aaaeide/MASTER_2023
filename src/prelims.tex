\chapter{Preliminaries}
\label{chap:prelims}
\epigraph{For simplicity, we also assume that every point in a geometry is a closed set. Without this additional assumption, the resulting structure is often described by the ineffably cacaphonic term "matroid", which we prefer to avoid in favor of the term "pregeometry".}{Gian-Carlo Rota \cite{crapo_rota_1970}}

Matroids were first introduced by Hassler Whitney in 1935~\cite{whitney-1935}, in a seminal paper where he described two axioms for independence in the columns of a matrix, and defined any system obeying these axioms to be a ``matroid'' (which unfortunately for Rota is the term that has stuck). Whitney's key insight was that this abstraction of~~``independence'' is applicable to both matrices and graphs. Matroids have also received attention from researchers in fair allocation, as their properties make them useful for modeling user preferences. We have already seen this with the course allocation problem described in the previous chapter; other use cases include the assignment of kindergarten slots or public housing estates among people of different ethnicities~\cite{benabbou-2021}.

\section{Fair allocation}
To ease readability, I abuse notation a bit and use $S+e$ and $S-e$ to refer to $S \cup \{e\}$ and $S \setminus \{e\}$, respectively.

An instance of a fair allocation problem consists of a set of agents $N = \{1,2,\ldots, n\}$ and a set of $m$ goods $E = \{e_1, e_2, \dots, e_m\}$. Each agent has a valuation function $v_i: 2^E \to \mathbb{R}^+$; $v_i(S)$ is the value agent $i$ ascribes to the bundle of goods $S$. The marginal value of agent $i$ for the good $g$, given that she already owns the bundle $S$, is given by $\Delta_i(S, g) := v_i(S + g) - v_i(S)$. Throughout most of this thesis, we assume that $v_i$ is a matroid rank function, or, equivalently, a binary submodular function. To formalize the description given in Chapter~1, this means that
\begin{enumerate}
  \item[(a)] $v_i(\emptyset) = 0$,
  \item[(b)] $v_i$ has binary marginals: $\Delta_i(A, g)\in \{0,1\}$ for every $A \subset E$ and $g\in E$,
  \item[(c)] $v_i$ is submodular: for every $A\subseteq B\subseteq E$ and $g\in E\setminus B$, we have that $\Delta_i(A, g) \geq \Delta_i(B, g)$.
\end{enumerate}
Any function $v_i$ adhering to these properties is a valid characterization of exactly one matroid in terms of its rank function~\cite{schrijver-2003}. There are many other ways to characterize matroids, some of which are given in Section~\ref{sec:matroid-theory}.

Throughout this thesis, I will use the terms \textit{allocation} and \textit{partition} somewhat interchangeably. In set theory, an $n$-partition of a set is a grouping of its elements into $n$ subsets, such that each element occurs in exactly one subset. It is often required that the subsets be non-empty---when working with matroids this requirement is usually omitted since the empty set is independent (see Section~\ref{sec:matroid-theory}). In a fair allocation instance with $n$ agents, an allocation is an $n$-partition of $E$. 

The output of an algorithm for fair allocation is an allocation of the goods to the agents. An allocation $A$ is an $n$-partition of $E$, $A = (A_1, A_2, \dots, A_n)$, where each $A_i$ is the bundle of goods allocated to agent $i$. Sharing is not allowed, so we require that $A_i\cap A_j = \emptyset$ for all $i\neq j$. We say that an allocation is \textit{clean} (also known as \textit{non-redundant} in the literature) if no agent has received any good they value at 0. An allocation is \textit{complete} if all goods are allocated, if not it is \textit{partially}. It might not be possible to guarantee both cleanness and completeness; for instance in a case where a good is 0-valued by all agents.

\subsection{Envy-freeness}
We are interested in producing \textit{fair} allocations. One of the most popular notions of fairness in the literature is envy-freeness (EF), which states that no agent should prefer another agent's bundle over her own (in fair allocation, an agent is \textit{envious} of another agent if she prefers that agent's bundle). An allocation $A$ is EF iff, for all agents $i,j\in N$,
\begin{equation} \tag{EF}
  v_i(A_i) \geq v_i(A_j).
\end{equation}

Because, as mentioned in the introduction, EF is not always achievable when the goods are indivisible, the literature has focused on relaxations thereof. The most prominent such relaxation, which can be guaranteed, is \textit{envy-freeness up to one good} (EF1)~\cite{lipton-2004}, which allows for the envy of up to the value of one (highest-valued) good. This is equivalent to saying that any envy can be eliminated by dropping one good from the envied bundle. $A$ is an EF1 allocation iff, for all agents $i,j \in N$ where $|A_j| > 0$, there exists a $e \in A_j$ such that
\begin{equation} \tag{EF1}
  v_i(A_i) \geq v_i(A_j - e).
\end{equation}

\textit{Envy-freeness up to any good} (EFX) is an even stronger version of EF. While EF1 allows that agent $i$ envies agent $j$ up to their highest valued good, EFX requires that the envy can be removed by dropping agent $j$'s least valued good. There are two slightly different definitions of EFX in use in the literature. I follow the naming scheme used by Benabbou et al.~\cite{benabbou-2021} and refer to these as EFX$_+$ and EFX$_0$. Caragiannis et al.~\cite{caragiannis-Unreasonable} requires that this least valued good be positively valued. We call this fairness objective EFX$_+$. $A$ is an EFX$_+$ allocation iff, for all agents $i,j \in N$,
\begin{equation} \tag{EFX$_+$}
  v_i(A_i) \geq v_i(A_j - e),\ \forall e \in A_j \text{ st. } v_i(A_j-e) < v_i(A_j).
\end{equation}
Plaut and Roughgarden~\cite{plaut2017envyfreeness}, on the other hand, allow for 0-valued goods in the envy check -- we call this version EFX$_0$. It is stronger requirement than EFX$_+$. $A$ is an EFX$_0$ allocation iff, for all agents $i,j \in N$,
\begin{equation} \tag{EFX$_0$}
  v_i(A_i) \geq v_i(A_j - e),\ \forall e \in A_j.
\end{equation}
In the general, additive case, the existence of EFX$_0$ allocations is an open question for instances with $n\geq4$~\cite{amanatidis2022fair}.

\subsection{Proportionality}
\textit{Proportionality} is a fairness objective that is fundamentally different from envy-freeness, in that it checks each bundle value against some threshold, instead of comparing bundle values against each other. An allocation $A$ is proportional (PROP) iff each agent $i\in N$ receives at least her proportional share PROP$_i$, which is the $\frac{1}{n}$ fraction of the value she puts on the whole set of goods, ie.
\begin{equation}\tag{PROP}
  v_i(A_i)\geq \text{PROP}_i := \frac{v_i(E)}{n}.
\end{equation}

Proportionality might not be achievable in the indivisible case (again, consider two agents and one positively valued good), and so relaxations in the same vein as EF1 and EFX have been introduced -- these are called PROP1 and PROPX~\cite{amanatidis2022fair}. An allocation is PROP1 iff there for each agent $i \in N$ exists some good $g\in E\setminus A_i$ that, if given to $i$, would ensure that agent $i$ received her proportional share; that is, 
\begin{equation}\tag{PROP1}
  \exists e\in E\setminus A_i\text{ st. }v_i(A_i+e)\geq\text{PROP}_i
\end{equation}

PROPX is a stronger fairness objective than PROP1, and has, as in the case of EFX, two slightly different definitions in the literature. I follow the naming scheme established for EFX above, and refer to these as PROPX$_+$ and PROPX$_0$. The logic is similar to that of EFX. An allocation is PROPX$_0$ iff each agent can achieve her proportional share by receiving one additional, least-valued good from the goods not allocated to her. This good might be zero-valued.
\begin{equation}\tag{PROPX$_0$}
  \min_{e\in E\setminus A_i}v_i(A_i + e)\geq\text{PROP}_i
\end{equation}
If we disallow zero-valued items, we arrive at the slightly weaker criteria PROPX$_+$, given by:
\begin{equation}\tag{PROPX$_+$}
  \min_{e\in E\setminus A_i, \Delta_i(A_i, e)>0}v_i(A_i + e)\geq\text{PROP}_i
\end{equation}

\subsubsection*{Maximin share fairness}

Budish introduces a relaxation of proportionality known as \textit{maximin share fairness}~\cite{Budish2011}, in which the threshold for each agent $i$ is her \textit{maximin share} (MMS). The MMS of agent $i$, denoted by $\mu_i$, is defined as the maximum value she could receive if she partitioned $E$ among all agents and then picked the worst bundle. Let $\Pi_n(E)$ be the family of all possible allocations of the goods in $E$ to the agents in $N$. Then,
$$\mu_i := \max_{A\in \Pi_n(E)} \min_{A_j \in A} v_i(A_j).$$
An allocation is MMS-fair if all agents receive at least as much as their maximin share:
\begin{equation} \tag{MMS}
  v_i(A_i) \geq \mu_i,\ \forall i \in N.
\end{equation}
In the general, additive case, MMS-fair allocations do not always exist, and even computing the MMS of an agent is an NP-hard problem~\cite{amanatidis2022fair}. In a setting with matroid-rank valuations, however, Barman and Verma showed that MMS-fair allocations always exist, and can be computed in polynomial time~\cite{barman2021existence}.

\subsection{Efficiency}
Fairness is usually coupled with some efficiency criterion, to prevent the perfectly fair solution in which the whole set of goods is thrown away. The efficiency of an allocation can be measured with some \textit{welfare function} on the values of the agents. There are three welfare functions commonly used in the literature:
\begin{enumerate}
  \item \textbf{Egalitarian social welfare (ESW):} The ESW of an allocation $A$ is given by the minimum value of an agent. ESW$(A) = \min_{i\in N}v_i(A_i)$.
  \item \textbf{Utilitarian social welfare (USW):} The USW of an allocation is the total value received by all agents. USW$(A) = \sum_{i\in N}v_i(A_i)$.
  \item \textbf{Nash welfare (NW):} The Nash welfare of an allocation is a compromise between the utilitarian and egalitarian approaches, given by the product of agent utilities. NW$(A) = \prod_{i\in N}v_i(A_i)$.
\end{enumerate}
Allocations that maximize one of these welfare functions are referred to as MAX-ESW, MAX-USW and MNW, respectively. 

\subsubsection*{Leximin}
An obvious drawback of the egalitarian approach is that, in the situation where there are multiple possible allocations that maximize the minimum bundle value, it is indifferent to which of these bundles it prefers. Consider for instance two possible allocations of goods to three agents, where their bundle values are given as (A,B,C):
\begin{itemize}
  \item Allocation 1: $(5,7,10)$
  \item Allocation 2: $(5,8,9)$
\end{itemize}
Both of these allocations are MAX-ESW. A stricter version of the egalitarian rule is \textit{leximin}: an allocation is leximin if it maximizes the smallest value; subject to that, it maximizes the second-smallest value; subject to that, it maximizes the next-smallest value, and so on. The leximin rule prefers Allocation 2 over Allocation 1.

\subsubsection*{Pareto Optimality}
\skelpar


\section{Matroid theory}
\label{sec:matroid-theory}
If a mathematical structure can be defined or axiomatized in multiple different, but not obviously equivalent, ways, the different definitions or axiomatizations of that structure make up a cryptomorphism. The many obtusely equivalent definitions of a matroid are a classic example of cryptomorphism, and belie the fact that the matroid is a generalization of concepts in many, seemingly disparate areas of mathematics. As a result, the terms used in matroid theory are borrowed from analogous concepts in both graph theory and linear algebra. 

The most common way to characterize a matroid is as an \textit{independence system}. An independence system is a pair $(E, \mathcal{I})$, where $E$ is the ground set of elements, $E \not= \emptyset$, and $\mathcal{I}$ is the set of independent sets, $\mathcal{I} \subseteq 2^E$. The \textit{dependent sets} of a matroid are $2^E \setminus \mathcal{I}$. 

In practice, the ground set $E$ represents the universe of elements in play, and the independent sets of typically represent the legal combinations of these items. In the context of fair allocation, the independent sets represent the legal (in the case of matroid constraints) or desired (in the case of matroid valuations) bundles of items.

A matroid is an independence system with the following properties~\cite{whitney-1935}:
\begin{enumerate}
  \item[(1)] If $S \subseteq T$ and $T \in \mathcal{I}$, then $S \in \mathcal{I}$.
  \item[(2)] If $S, T \in \mathcal{I}$ and $|S| > |T|,$ then there exists $e \in S \setminus T$ such that $S + e \in S$.
  \item[(2')] If $S \subseteq E$, then the maximal independent subsets of $S$ are equal in size.
\end{enumerate}
Property (1) is called the \textit{hereditary property} and (2) the \textit{exchange property}. Properties (2) and (2') are equivalent. To see that $(2) \implies$~(2'), consider two maximal subsets of $S$. If they differ in size, (2) tells us that there are elements we can add from one to the other until they have equal cardinality. We get (2')~$\implies (2)$ by considering $S = A \cup B$. Since $|A|>|B|$, they cannot both be maximal, and some $e \in A \setminus B$ can be added to $B$ to obtain another independent set. 

When $S=E$, (2') gives us that the maximal independent sets of a matroid are all of the same size. A maximal independent subset of $E$ is known as a \textit{basis}. The size of the bases is the \textit{rank} of the matroid as a whole.  A matroid can be exactly determined by $\mathcal{B}$, its collection of bases, since a set is independent if and only if it is contained in a basis (this follows from (1)). A theorem by Whitney~\cite{whitney-1935} gives the axiom system characterizing a collection of bases of a matroid: 
\begin{enumerate}
  \item No proper subset of a basis is a basis.
  \item If $B, B'\in \mathcal{B}$ and $e \in B$, then for some $e'\in B'$, $B-e+e'\in\mathcal{B}$.
\end{enumerate} 

The rank function of a matroid is a function $v:2^E \to \mathbb{Z}^+$ which, given a subset $B\subseteq E$, returns the size of the largest independent set contained in $B$. That is, $$v(B) = \max_{A \subseteq B, A\in \mathcal{I}}|A|.$$

\subsection{A graphic example}
Different types of matroids exist, arising from different sources of ``independence''; one well-known subclass of matroids, arising from notions of independence in graphs, is the class of \textit{graphic matroids}.

A \textit{tree} is a connected acyclic graph, and a \textit{forest} is a disconnected graph consisting of some number of trees. A \textit{spanning tree} of $G$ is a subgraph with a unique simple path between all pairs of vertices of $G$. A \textit{spanning forest} of $G$ is a collection of spanning trees, one for each component. A graph will have some number of different spanning trees. Figure~\ref{fig:ex-graph-mst} shows two spanning trees of the same graphs (or alternatively, a spanning forest over one graph with two components).

Given a graph $G=(V,E)$, let $\mathcal{I} \subseteq 2^E$ be the family of subsets of the edges $E$ such that, for each $I \in \mathcal{I},\ (V, I)$ is a forest. It is a classic result of matroid theory that $\mathfrak{M} = (E, \mathcal{I})$ (the ground set of the matroid being the edges of the graph) is a matroid~\cite[p.~657]{schrijver-2003}. To understand how, we will show that it adhers to axioms (1) and (2'), as given in above. By investigating the highlighted spanning trees in Figure~\ref{fig:ex-graph-mst}, it is easy to convince oneself that all subsets of a spanning tree are trees, since no subset of an acyclic set of edges will contain a cycle. Thus axiom (1) -- the hereditary property -- holds. 

To see that (2') holds, consider the set of bases of the matroid, $\mathcal{B} \subseteq \mathcal{I}$. Figure~\ref{fig:ex-graph-mst} shows two bases of the matroid described by the graph. By definition, each basis $B \in \mathcal{B}$ is a maximal forest over $G$. Since a spanning tree of a graph with $n$ nodes must needs have $n-1$ edges (I recommend drawing trees and counting their edges until one is convinced that this must be the case), we have $|B| = |V| - k$, where $k$ is the number of components of $G$. This is the same for every $B \in \mathcal{B}$, which proves property (2'). Any matroid given by a graph $G$, denoted by $\mathfrak{M}(G)$, is called a graphic matroid.

\begin{figure}
  \centering
  \begin{minipage}{.4\textwidth}
    \begin{tikzpicture}[scale=0.75]
      \node[draw,circle, inner sep=3pt] (q) at (0,0)   {};
      \node[draw,circle, inner sep=3pt] (w) at (-3,-2) {};
      \node[draw,circle, inner sep=3pt] (e) at (0,-2)  {};
      \node[draw,circle, inner sep=3pt] (r) at (3,-2)  {};
      \node[draw,circle, inner sep=3pt] (t) at (-2,-4) {};
      \node[draw,circle, inner sep=3pt] (y) at (2,-4)  {};
      
      \draw[preaction={draw=darktan, line width=2mm}] (q) -- (w); 
      \draw[preaction={draw=darktan, line width=2mm}] (q) -- (e); 
      \draw[preaction={draw=darktan, line width=2mm}] (q) -- (r); 
      \draw (w) -- (t); 
      \draw[preaction={draw=darktan, line width=2mm}] (e) -- (t); 
      \draw[preaction={draw=darktan, line width=2mm}] (e) -- (y); 
      \draw (r) -- (y); 
      \draw (t) -- (y); 
    \end{tikzpicture}
  \end{minipage}
  \hspace{0.75cm}
  \begin{minipage}{.4\textwidth}
    \begin{tikzpicture}[scale=0.75]
      \node[draw,circle, inner sep=3pt] (q) at (0,0)   {};
      \node[draw,circle, inner sep=3pt] (w) at (-3,-2) {};
      \node[draw,circle, inner sep=3pt] (e) at (0,-2)  {};
      \node[draw,circle, inner sep=3pt] (r) at (3,-2)  {};
      \node[draw,circle, inner sep=3pt] (t) at (-2,-4) {};
      \node[draw,circle, inner sep=3pt] (y) at (2,-4)  {};
      
      \draw (q) -- (w); 
      \draw[preaction={draw=darktan, line width=2mm}] (q) -- (e); 
      \draw (q) -- (r); 
      \draw[preaction={draw=darktan, line width=2mm}] (w) -- (t); 
      \draw[preaction={draw=darktan, line width=2mm}] (e) -- (t); 
      \draw (e) -- (y); 
      \draw[preaction={draw=darktan, line width=2mm}] (r) -- (y); 
      \draw[preaction={draw=darktan, line width=2mm}] (t) -- (y); 
    \end{tikzpicture}
  \end{minipage}

  \caption{Two spanning trees of a graph with 8 edges.}
  \label{fig:ex-graph-mst}
\end{figure}
  


\subsection{Other characterizations of a matroid} 
\label{sec:characterizations}

We have already seen the independent sets characterization of a matroid, but the other properties of a matroid have their own axiom systems that can equivalently be used to characterize a matroid.

\paragraph{Characterization via rank function.} \skelpar

\paragraph{Characterization via circuits.} A \textit{circuit} is a minimal dependent set of a matroid -- it is an independent set plus one redundant element. Equivalently, the collection of circuits of a matroid is given by
$$\mathcal{C} = \bigl\{ C : |C| = v(C) + 1, C\subseteq E \bigr\}.$$
A set is independent if and only if it contains no circuit~\cite{schrijver-2003}, and so a matroid is uniquely determined by the collection of its circuits. The following conditions characterize $\mathcal{C}$~\cite{whitney-1935}:
\begin{enumerate}
  \item[(1)] No proper subset of a circuit is a circuit.
  \item[(2)] If $C, C'\in\mathcal{C}$, $x\in C\cap C'$ and $y\in C\setminus C'$, then $C\cup C'$ contains a circuit containing $y$ but not $x$.
\end{enumerate}

\begin{figure}
  \centering
  \begin{tikzpicture}
    \node[draw,circle, inner sep=3pt] (q) at (0,0)   {};
    \node[draw,circle, inner sep=3pt] (w) at (-3,-2) {};
    \node[draw,circle, inner sep=3pt] (e) at (0,-2)  {};
    \node[draw,circle, inner sep=3pt] (r) at (3,-2)  {};
    \node[draw,circle, inner sep=3pt] (t) at (-2,-4) {};
    \node[draw,circle, inner sep=3pt] (y) at (2,-4)  {};
    
    \draw[preaction={draw=darktan, line width=2mm}]           (q) -- (w); 
    \draw[preaction={draw=averagegreen, line width=2mm}]      (q) -- (e); 
    \draw[preaction={draw=complementaryblue, line width=2mm}] (q) -- (r); 
    \draw[preaction={draw=darktan, line width=2mm}] (w) -- (t); 
    \draw[preaction={draw=darktan, line width=2mm}] (e) -- (t); 
    \draw[preaction={draw=complementaryblue, line width=2mm}] (e) -- (y); 
    \draw[preaction={draw=complementaryblue, line width=2mm}] (r) -- (y); 
    \draw (t) -- (y); 
  \end{tikzpicture}
  \caption{Two overlapping circuits on a graph.}
  \label{fig:ex-graph-circuits}
\end{figure}

These properties are easily grokked by studying an example. In Figure~\ref{fig:ex-graph-circuits}, we see two circuits of a graph, highlighted in yellow and blue, overlapping at the edge highlighted in green. Obviously, no circuit in this figure contains a circuit. We can also verify that the union of the yellow edge set and the blue edge set, minus the green edge, is in fact a third, bigger circuit, as promised by property (2).

\paragraph{Characterization via closed sets.} We also need to establish the concept of the \textit{closed sets} (sometimes referred to as \textit{flats}~\cite{schrijver-2003}) of a matroid. A closed set is a set whose cardinality is maximal for its rank. Equivalently to the definition given above, we can define a matroid as $\mathfrak{M} = (E, \mathcal{F})$, where $\mathcal{F}$ is the set of closed sets of $\mathfrak{M}$, satisfying the following properties~\cite{knuth-1975}:

\begin{enumerate}
  \item The set of all elements is closed: $E \in \mathcal{F}$
  \item The intersection of two closed sets is a closed set: If $A,B \in \mathcal{F},$ then $A \cap B \in \mathcal{F}$
  \item If $A \in \mathcal{F}$ and $a,b \in E \setminus A,$ then $b$ is a member of all sets in $\mathcal{F}$ containing $A \cup \{a\}$ if and only if $a$ is a member of all sets in $\mathcal{F}$ containing $A \cup \{b\}$
\end{enumerate}

The \textit{closure function} is the function $\fn{cl} : 2^E \to 2^E$, such that $$\fn{cl}(S) = \bigl\{ x \in E : \fn{v}(S) = \fn{v}(S \cup \{x\}) \bigr\}.$$ That is to say, the closure function, when given a set $S \subseteq E$, returns the set of elements in $x \in E$ such that $x$ can be added to $S$ with no increase in rank. It returns the closed set of the same rank as $S$, that contains $S$. The \textit{nullity} of a subset $S$ is the difference $|S| - v(S)$, ie. the number of elements that must be removed from $S$ to obtain an independent set.

Figure~\ref{fig:ex-graph-closure} shows an independent (acyclic) set $S$ of edges in highlighted yellow, along with its closure $\fn{cl}(S)$ in blue. The closure is the set of edges $e$ such that $v(S+e) = v(S)$, or, equivalently, such that the spanning tree of $S+e$ has the same size as that of $S$.

\begin{figure}
  \centering
  \begin{tikzpicture}[every loop/.style={}]
    \node[draw,circle,inner sep=3pt] (q) at (0,0) {};
    \node[draw,circle,inner sep=3pt] (w) at (2,0) {};
    \node[draw,circle,inner sep=3pt] (e) at (-1,-1) {};
    \node[draw,circle,inner sep=3pt] (r) at (0,-2) {};
    \node[draw,circle,inner sep=3pt] (t) at (2,-2) {};
    \node[draw,circle,inner sep=3pt] (y) at (4, 0) {};
    \node[draw,circle,inner sep=3pt] (u) at (4,-2) {};
    \node[draw,circle,inner sep=3pt] (i) at (6,-2) {};
    
    \begin{scope}[on background layer]
      \draw[line width=2mm, complementaryblue] (t) to [bend left] (w);
    \end{scope}
    
    \path (q) edge [bend left, preaction={draw=darktan, line width=2mm}] node {} (w);
    \path (q) edge [bend right, preaction={draw=complementaryblue, line width=2mm}] node {} (w);
    \draw (q) -- (e);
    \draw[preaction={draw=darktan, line width=2mm}] (q) -- (r);
    \path (e) edge [loop left, preaction={draw=complementaryblue, line width=2mm}] node {} (e);
    \draw (e) -- (r);
    \draw[preaction={draw=darktan, line width=2mm}] (r) -- (t);
    \path (t) edge [bend left] node {} (w);
    \path (t) edge [bend right, preaction={draw=complementaryblue, line width=2mm}] node {} (w);
    \draw[preaction={draw=darktan, line width=2mm}] (y) -- (u);
    \draw (u) -- (i);
  \end{tikzpicture}
  \caption{An independent subset of edges in yellow, with its closure in blue.}
  \label{fig:ex-graph-closure}
\end{figure}

\subsection{Matroid union}
\label{sec:matroid-union}
The matroid union is an operation that allows us to produce a new matroid by combining the independent sets of a collection of existing matroids. This is a powerful tool, as it allows us to reason about independence across multiple matroids, and has found several practical applications within fair allocation.

Given $n$ matroids $\mathfrak{M}_i = (E, \mathcal{I}_i)$, their union is given, somewhat obtusely, by $$\widehat{\mathfrak{M}} = (E, \widehat{\mathcal{I}}) = (E, \{ I_1\cup\ldots\cup I_n : I_i\in\mathcal{I}_i,\ \forall i\in N \}).$$ 
$\widehat{\mathfrak{M}}$ is in fact a matroid~\cite[Ch. 42]{schrijver-2003}, whose independent sets are the the subsets $S\subseteq E$ that allow an $n$-partition $S_1,\ldots,S_n$ such that $S_i\in\mathcal{I}_i$, for all $i\in N$. The following statements are equivalent for all $n$-partitions of the elements of $E$:
\begin{enumerate}
  \item Each $S_i\in S$ is independent in $\mathfrak{M}_i$
  \item $S=(S_1,\dots,S_n)\in\mathcal{I}_i\times\dots\times\mathcal{I}_n$
  \item $\bigcup_{i=1}^n S_i \in \widehat{\mathcal{I}}$ is independent in $\widehat{\mathfrak{M}}$
\end{enumerate}

In fair allocation jargon, if each $\mathfrak{M}_i$ is the matroid described by agent $i$'s valuation function $v_i$, a set of goods $S$ is independent in $\widehat{\mathfrak{M}}$ if and only if we can allocate it among the agents and produce utilitarian social welfare (ie. total value) equal to $|S|$. This follows from the fact that if $\mathcal{S}$ is an $n$-partition of $S$ such that $\mathcal{S} = (S_1,\dots,S_n)\in\mathcal{I}_1\times\dots\times\mathcal{I}_n$, then $S_i\in\mathcal{I}_i$ for each $i$. When this is the case, we have $v_i(S_i) = |S_i|$, and so SW$(S) = \sum_{i\in N}v_i(S_i) = \sum_{i\in N}|S_i| = |S|$.

Hence, each basis in $\widehat{\mathfrak{M}}$ corresponds to a clean (but not necessarily complete), MAX-USW allocation of the goods in $E$~\cite{barman2021existence}. It is a classic result of Edmonds~\cite{Edmonds2009} that a basis in $\widehat{\mathfrak{M}}$ can be computed in polynomial time, using the \textit{matroid union algorithm}. This is achieved by making use of two additional concepts, the \textit{exchange graph} and \textit{transfer paths}. 

\subsubsection*{The exchange graph}
Given $n$ matroids $\mathfrak{M}_1 = (E, \mathcal{I}_1),\ldots,\mathfrak{M}_n = (E,\mathcal{I}_n)$, let $A$ be a collection of $n$ sets $A_1,\dots,A_n$ such that (a) $A_i\cap A_j = \emptyset$ when $i\neq j$, and (b) for each $i$, $A_i\subseteq E$ and $A_i\in\mathcal{I}_i$. In the context of a matroid-rank-valued fair allocation problem, $A$ is a clean allocation of the goods in $E$ to the agents in $N$.

We follow the example of Schrijver~\cite{schrijver-2003} and define the exchange graph of $A$ as the directed graph $D(A)=(E, x(A))$, where each node corresponds to a good in $E$, and the edges are given by
$$x(A) = \{ (p,q) : p \in A_i, q \in E\setminus A_i, v_i(A_i - p + q) = v_i(A_i) \},$$
where $v_i$ is the rank function of the matroid $\mathfrak{M}_i$. In other words, an edge exists between goods $p\in A_i$ and $q\in E\setminus A_i$ iff we can replace $p$ with $q$ for no decrease in the rank of the set containing $p$. Intuitively, we can understand $D$ as representing for each good $p$, which other good $q$ the current owner of $p$ can replace $p$ with and be just as happy. This intuitive explanation lets us begin to see why the matroid union algorithm and the concept of the exchange graph has found widespread use in fair allocation with matroid-rank valuations; they allow us to model equitable transfers of goods between agents.

\subsubsection*{Transfer paths and path augmentation}
Let $P = (e_1, \dots, e_t)$ be a path in the exchange graph $D(A)$. The \textit{transfer} of goods along $P$ is the operation in which $e_t$ is given to the agent who owns $e_{t-1}$, $e_{t-1}$ is given to the agent who owns $e_{t-2}$, and so on until $e_1$ is discarded. This transfer is called \textit{path augmentation}; we use the notation established by Viswanathan and Zick~\cite{viswanathan2023yankee} and denote the bundle $A_i$ after augmentation with the path $P$ by $A_i \Lambda P$. 

For some $i\in N$, we define $F_i = \{ e\in E\setminus A_i : A_i + e \in \mathcal{I}_i \}$ as the set of elements whose addition to $A_i$ yields another, larger independent set (remember that $A$ is clean, so $A_i\in\mathcal{I}_i$ for each $i$). In a paper on the matroid union algorithm, Knuth~\cite{knuth1973matroidpartitioning} shows that, by augmenting along a shortest path $P = (e_1,\dots,e_t)$ from $F_i$ to $A_j$ for some $j \in N - i$, we get
\begin{enumerate}
  \item[(a)] $v_i(A_i\Lambda P + e_1) = v_i(A_i) + 1$,
  \item[(b)] $v_k(A_k\Lambda P) = v_k(A_k),\ \forall k\in N - i - j$, and
  \item[(c)] $v_j(A_j\Lambda P) = v_j(A_j) - 1$.
\end{enumerate}
By greedily growing clean bundles in this manner, we can find a maximal independent set over $\widehat{\mathfrak{M}}$ in polynomial time~\cite{schrijver-2003}. Chapter~\ref{chap:matroids.jl} discusses Matroids.jl implementation of Knuth's matroid union algorithm. Viswanathan and Zick's Yankee Swap algorithm~\cite{viswanathan2023yankee}, discussed in Chapter~\ref{chap:yankee-swap}, uses these concepts as well.

\subsection{Matroid erection}
Knuth's general matroid construction, the implementation of which is discussed in Section~\ref{sec:kmc}, requires us to establish a few more matroid concepts. The \textit{rank-k truncation} of a rank-$r$ matroid $\mathfrak{M} = (E, \mathcal{I})$, is the rank-$k$ matroid $\mathfrak{M}^{(k)} = (E, \mathcal{I}^{(k)})$, where
$$\mathcal{I}^{(k)} = \{ I \in \mathcal{I} : |I| \leq k \}.$$
The \textit{truncation} of $\mathfrak{M}$ is given as $T(\mathfrak{M}) = \mathfrak{M}^{(r-1)}$. As a simple example, we have that the uniform matroid $U_n^{n-1} = T(F_n)$, where $F_n$ is the free matroid with $n$ elements. The \textit{erection} (I defer to Crapo for the somewhat esoteric choice of term~\cite{Crapo1970}) of the matroid $\mathfrak{M}$ is the matroid $\mathfrak{N}$ of rank $\leq r+1$ such that $\mathfrak{M} = T(\mathfrak{N})$. In other words, when taking the truncation of a matroid, we produce a new matroid by disregarding all independent sets of cardinality higher than some $k$. When taking the erection of a rank-$k$ matroid, we produce a new matroid by declaring some number of subsets $S \subseteq E$ such that $|S|=r+1$ to be independent, or equivalently, declaring some number of them to be closed sets of rank $r$. By declaring all of them to be dependent, we get the \textit{trivial erection} $\mathfrak{N} = \mathfrak{M}$. By declaring none of them dependent (ie. they are all independent sets $r(S) = |S| = r+1$), $\mathfrak{N}$ is the \textit{free erection} of $\mathfrak{M}$~\cite{greene-1991}.

\begin{figure}\centering
  \resizebox{\columnwidth}{!}{
         \begin{tikzpicture}[> = stealth,  shorten > = 1pt,   auto,   node distance = 1.5cm, mynode/.style={pattern=north east lines, circle, draw,inner sep=2pt,outer sep=0pt}]
  
  %%% LEVEL 1
  \node[draw, mynode]  (x) {1111};
  %%% LEVEL 2
  \node[draw, mynode]  [below left of=x] (B) {1101};
  \node[draw, mynode] [below right of=x]  (C) {1011};
  \node[draw, mynode]  [left  of=B] (A) {1110};
  \node[draw, mynode] [right of=C]  (D) {0111};
  %%%% LEVEL 3
  \node[draw, mynode, fill=darktan] [below of=B]  (AD) {0110};
  \node[draw, mynode, fill=darktan] [below of=C]  (BC) {1001};
  \node[draw, mynode, fill=darktan] [below of=D]  (BD) {0101};
  \node[draw, mynode, fill=darktan] [below of=A]  (AC) {1010};
  \node[draw, mynode, fill=darktan] [left  of=AC] (AB) {1100};
  \node[draw, mynode, fill=darktan] [right of=BD] (CD) {0011};
  
  %%  LEVEL   4
  \node[draw, mynode, fill=darktan] [below of=AD]  (ABD) {0100};
  \node[draw, mynode, fill=darktan] [below of=BC]  (ACD) {0010};
  \node[draw, mynode, fill=darktan] [left of=ABD]  (ABC) {1000};
  \node[draw, mynode, fill=darktan] [right of=ACD] (BCD) {0001};
  
  %%%% LEVEL 5
  \node[draw, mynode, fill=darktan] [below right of=ABD]  (ABCD) {0000};

  \begin{scope}[on background layer]
    \draw[line width=2mm, darktan] (AB) to (ABC);
    \draw[line width=2mm, darktan] (AB) to (ABD);
    \draw[line width=2mm, darktan] (AC) to (ABC);
    \draw[line width=2mm, darktan] (AC) to (ACD);
    \draw[line width=2mm, darktan] (AD) to (ABD);
    \draw[line width=2mm, darktan] (AD) to (ACD);
    \draw[line width=2mm, darktan] (BC) to (ABC);
    \draw[line width=2mm, darktan] (BC) to (BCD);
    \draw[line width=2mm, darktan] (BD) to (ABD);
    \draw[line width=2mm, darktan] (BD) to (BCD);
    \draw[line width=2mm, darktan] (CD) to (ACD);
    \draw[line width=2mm, darktan] (CD) to (BCD);
  \end{scope}
  
  
  %% LEVEL 1 to LEVEL 2
      \path (x)  edge node {} (A);
      \path (x)  edge node {} (B);
      \path (x)  edge node {} (C);
      \path (x)  edge node {} (D);
  
  %% LEVEL 2 to LEVEL 3
      \path (A)  edge node {} (AB);
      \path (A)  edge node {} (AC);
      \path (A)  edge node {} (AD);
      \path (B)  edge node {} (AB);
      \path (B)  edge node {} (BC);
      \path (B)  edge node {} (BD);
      \path (C)  edge node {} (AC);%
      \path (C)  edge node {} (BC);
      \path (C)  edge node {} (CD);
      \path (D)  edge node {} (AD);
      \path (D)  edge node {} (BD);
      \path (D)  edge node {} (CD);
  %%% LEVEL 3 TO 4
  
      \path (AB)  edge node {} (ABC);
      \path (AB)  edge node {} (ABD);
      \path (AC)  edge node {} (ABC);
      \path (AC)  edge node {} (ACD);
      \path (AD)  edge node {} (ABD);
      \path (AD)  edge node {} (ACD);
      \path (BC)  edge node {} (ABC);
      \path (BC)  edge node {} (BCD);
      \path (BD)  edge node {} (ABD);
      \path (BD)  edge node {} (BCD);
      \path (CD)  edge node {} (ACD);
      \path (CD)  edge node {} (BCD);
  %%%% LEVEL 4 to 5
      \path (ABC)  edge [preaction={draw=darktan, line width=2mm}] node {} (ABCD);
      \path (ABD)  edge [preaction={draw=darktan, line width=2mm}] node {} (ABCD);
      \path (ACD)  edge [preaction={draw=darktan, line width=2mm}] node {} (ABCD);
      \path (BCD)  edge [preaction={draw=darktan, line width=2mm}] node {} (ABCD);

      %%%% Dashed lines
% \draw[dashed] (-6,-0.5) -- (6,-0.5); 
% \node[] at (-6,0) {Rank 4};
% \draw[dashed] (-6,-1.75) -- (6,-1.75); 
% \node[] at (-6,-1) {Rank 3};
\draw[dashed] (-6,-3.5) -- (6,-3.5); 
\node[] at (-6,-2.625) {Rank 2};
\draw[dashed] (-6,-4.625) -- (6,-4.625); 
\node[] at (-6,-4) {Rank 1};
\node[] at (-6,-5) {Rank 0};

        \end{tikzpicture}
  }
  \caption{The uniform matroid $U_4^2$.}
  \label{fig:hasse-uniform}
\end{figure}

\begin{figure}\centering
  \resizebox{\columnwidth}{!}{
         \begin{tikzpicture}[> = stealth,  shorten > = 1pt,   auto,   node distance = 1.5cm, mynode/.style={pattern=north east lines, circle, draw,inner sep=2pt,outer sep=0pt}]
  
  %%% LEVEL 1
  \node[draw, mynode]  (x) {1111};
  %%% LEVEL 2
  \node[draw, mynode, fill=darktan]  [below left of=x] (B) {1101};
  \node[draw, mynode, fill=darktan] [below right of=x]  (C) {1011};
  \node[draw, mynode, fill=darktan]  [left  of=B] (A) {1110};
  \node[draw, mynode] [right of=C]  (D) {0111};
  %%%% LEVEL 3
  \node[draw, mynode, fill=darktan] [below of=B]  (AD) {0110};
  \node[draw, mynode, fill=darktan] [below of=C]  (BC) {1001};
  \node[draw, mynode, fill=darktan] [below of=D]  (BD) {0101};
  \node[draw, mynode, fill=darktan] [below of=A]  (AC) {1010};
  \node[draw, mynode, fill=darktan] [left  of=AC] (AB) {1100};
  \node[draw, mynode, fill=darktan] [right of=BD] (CD) {0011};
  
  %%  LEVEL   4
  \node[draw, mynode, fill=darktan] [below of=AD]  (ABD) {0100};
  \node[draw, mynode, fill=darktan] [below of=BC]  (ACD) {0010};
  \node[draw, mynode, fill=darktan] [left of=ABD]  (ABC) {1000};
  \node[draw, mynode, fill=darktan] [right of=ACD] (BCD) {0001};
  
  %%%% LEVEL 5
  \node[draw, mynode, fill=darktan] [below right of=ABD]  (ABCD) {0000};

  \begin{scope}[on background layer]
    \draw[line width=2mm, darktan] (A) to (AB);
    \draw[line width=2mm, darktan] (A) to (AC);
    \draw[line width=2mm, darktan] (A) to (AD);
    \draw[line width=2mm, darktan] (B) to (AB);
    \draw[line width=2mm, darktan] (B) to (BC);
    \draw[line width=2mm, darktan] (B) to (BD);
    \draw[line width=2mm, darktan] (C) to (AC);
    \draw[line width=2mm, darktan] (C) to (BC);
    \draw[line width=2mm, darktan] (C) to (CD);
    
    \draw[line width=2mm, darktan] (AB) to (ABC);
    \draw[line width=2mm, darktan] (AB) to (ABD);
    \draw[line width=2mm, darktan] (AC) to (ABC);
    \draw[line width=2mm, darktan] (AC) to (ACD);
    \draw[line width=2mm, darktan] (AD) to (ABD);
    \draw[line width=2mm, darktan] (AD) to (ACD);
    \draw[line width=2mm, darktan] (BC) to (ABC);
    \draw[line width=2mm, darktan] (BC) to (BCD);
    \draw[line width=2mm, darktan] (BD) to (ABD);
    \draw[line width=2mm, darktan] (BD) to (BCD);
    \draw[line width=2mm, darktan] (CD) to (ACD);
    \draw[line width=2mm, darktan] (CD) to (BCD);
  \end{scope}
  
  
  %% LEVEL 1 to LEVEL 2
      \path (x)  edge node {} (A);
      \path (x)  edge node {} (B);
      \path (x)  edge node {} (C);
      \path (x)  edge node {} (D);
  
  %% LEVEL 2 to LEVEL 3
      \path (A)  edge node {} (AB);
      \path (A)  edge node {} (AC);
      \path (A)  edge node {} (AD);
      \path (B)  edge node {} (AB);
      \path (B)  edge node {} (BC);
      \path (B)  edge node {} (BD);
      \path (C)  edge node {} (AC);%
      \path (C)  edge node {} (BC);
      \path (C)  edge node {} (CD);
      \path (D)  edge node {} (AD);
      \path (D)  edge node {} (BD);
      \path (D)  edge node {} (CD);
  %%% LEVEL 3 TO 4
  
      \path (AB)  edge node {} (ABC);
      \path (AB)  edge node {} (ABD);
      \path (AC)  edge node {} (ABC);
      \path (AC)  edge node {} (ACD);
      \path (AD)  edge node {} (ABD);
      \path (AD)  edge node {} (ACD);
      \path (BC)  edge node {} (ABC);
      \path (BC)  edge node {} (BCD);
      \path (BD)  edge node {} (ABD);
      \path (BD)  edge node {} (BCD);
      \path (CD)  edge node {} (ACD);
      \path (CD)  edge node {} (BCD);
  %%%% LEVEL 4 to 5
      \path (ABC)  edge [preaction={draw=darktan, line width=2mm}] node {} (ABCD);
      \path (ABD)  edge [preaction={draw=darktan, line width=2mm}] node {} (ABCD);
      \path (ACD)  edge [preaction={draw=darktan, line width=2mm}] node {} (ABCD);
      \path (BCD)  edge [preaction={draw=darktan, line width=2mm}] node {} (ABCD);
      
      %%%% Dashed lines
% \draw[dashed] (-6,-0.5) -- (6,-0.5); 
% \node[] at (-6,0) {Rank 4};
\draw[dashed] (-6,-1.75) -- (1.75,-1.75); 
\draw[dashed] (1.75,-1.75) -- (1.75,-0.25); 
\draw[dashed] (1.75,-0.25) -- (6,-0.25); 
\node[] at (-6,-1) {Rank 3};
\draw[dashed] (-6,-3.5) -- (6,-3.5); 
\node[] at (-6,-2.625) {Rank 2};
\draw[dashed] (-6,-4.625) -- (6,-4.625); 
\node[] at (-6,-4) {Rank 1};
\node[] at (-6,-5) {Rank 0};

        \end{tikzpicture}
  }
  \caption{An erection of $U_4^2$.}
  \label{fig:hasse-erection}
\end{figure}

A matroid can have many erections. Figure~\ref{fig:hasse-uniform} shows the Hasse diagram for the matroid of 4 elements, where every subset of two or fewer elements is independent (denoted in yellow). Each subset is encoded as a binary string, where a 1 designates membership in the set (a syntax we will become very familiar with in Chapter~\ref{chap:generating_matroids}). This matroid is known as the uniform matroid $U_4^2$. The uniform matroid $U_4^3$ is the free erection of $U_4^2$, where we have simply designated all sets of rank 3 as independent. Figure~\ref{fig:hasse-erection} shows another erection of $U_4^2$, one in which only three of the subsets with three elements are designated as independent. The final set, 0111, remains a dependent set, and has therefore in fact been designated as a closed set of rank $2$. By starting from the rank-0 matroid $\mathfrak{M}^{(0)}$, one can iteratively erect any matroid $\mathfrak{M}$, at each iteration $i$ designating the sets to be closed sets of rank $i$ (all the while ensuring that the axioms for the closed sets of a matroid are obeyed). This is the approach taken by Knuth in his matroid erection algorithm, which is discussed in Chapter~\ref{sec:kmc}.

The \textit{essential closed sets} (often referred to as essential flats) of a matroid are the closed sets whose existence cannot be inferred from the closed sets of lower rank. The rank-2 set 0111 in the example above is an essential closed set. The essential closed sets of a matroid, together with their ranks, fully determine the matroid~\cite{kelly-1978}. A rank-$r$ matroid can also be described inductively as a sequence of erections $(X_0, X_1, \dots, X_{r-1})$, where $X_k$ is a set of $(k+1)$-subsets of $E$ that roughly encode the closed sets of rank $k$~\cite{pendavingh2016enumerating}. This insight is the basis for Knuth's matroid construction algorithm.


\section{Matroids in fair allocation}
It should be clear at this point that matroids are compelling structures to work with in the context of fair allocations. There are two main use cases for matroids in fair allocation: matroid-rank valuation functions (as in the example scenario from the introduction) and matroid constraints. Matroids.jl will be developed with the empirical study of algorithms for these two scenarios in mind.

\subsection{Matroid-rank valuations}
In a fair allocation instance with matroid-rank valuations, each agent $i$ has a corresponding matroid $\mathfrak{M}_i = (\mathcal{M}, \mathcal{I}_i)$, where $\mathcal{M}$ (the set of goods) is the ground set of elements common to all agents' matroids.

Babaioff 2021 -- PE mechanism -- non-redundant Lorenz dominating allocation always exists and is MNW, MAX-USW, leximin, EFX (EF1) and .5-MMS. With a randomized prioritization it is ex ante EF and ex ante proportional as well. 3Yankee Swap computes prioritized non-redundant Lorenz dominating allocations.


Limitation: MRFs cannot model complementary goods!


\subsection{Matroid constraints}
Another usage found for matroids in fair allocation is that of \textit{matroid constraints}. The majority of work on fair division assumes that any allocation is feasible, and the sole concern is finding an allocation that aligns well with the agents' valuation profiles. In many practical applications, however, there will be allocations that are not legal or desirable. Suksompong~\cite{suksumpong-constraints} gives the example of a museum with multiple branches distributing exhibits of different categories (sculpture, paintings, et cetera) among the branches. For each category, it wants to create a balanced distribution among the branches, so that the difference in the number of exhibits of a given category differ by at most one between any branch. This is an example of a \textit{cardinality constraint}, which can be modeled with a partition matroid, and is a subset of the broader class of matroid constraints. 

When matroid constraints are enforced on a fair allocation instance, we require that all bundles be independent sets on some supplied matroid, common to all agents. 