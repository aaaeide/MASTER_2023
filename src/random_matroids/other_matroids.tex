\section{Basic types of matroids}
\skelpar

\subsection{Uniform matroids}
A uniform matroid $U_n^r$ is the matroid over $n$ elements where the independent sets are exactly the sets of cardinality at most $r$. The free matroid $U_n^n = (E, 2^E)$ is a special case of the uniform matroid and is the simplest and least interesting type of matroid, being the trivial case in which every subset of $E$ is an independent set. In Matroids.jl, we represent uniform matroids with a simple struct.

\begin{jllisting}
struct UniformMatroid
  n::Integer
  r::Integer
end

FreeMatroid(n) = UniformMatroid(n, n)
\end{jllisting}

\subsection{Linear matroids}

\subsection{Graphic matroids}

We begin with defining the graph theory terms used in this section. An undirected graph $G=(V,E)$ is said to be \textit{connected} if there exists at least one path between each pair of nodes in the graph; otherwise it is \textit{disconnected}. A disconnected graph consists of at least two connected subsets of nodes. These connected subgraphs are called \textit{components}. A \textit{tree} is a connected acyclic graph, and a \textit{forest} is a disconnected graph consisting of some number of trees. A \textit{spanning tree} of $G$ is a subgraph with a unique simple path between all pairs of vertices of $G$. A \textit{spanning forest} of $G$ is a collection of spanning trees, one for each component.

Given a graph $G=(V,E)$, let $\mathcal{I} \subseteq 2^E$ be the family of subsets of the edges $E$ such that, for each $I \in \mathcal{I},\ (V, I)$ is a forest. It is a classic result of matroid theory that $\mathfrak{M} = (E, \mathcal{I})$ is a matroid~\cite[p.~657]{schrijver-2003}. To understand how, we will show that it adhers to axioms (1) and (2'), as given in Section~\ref{sec:matroid-theory}. (1) holds trivially, as all subsets of a forest are forests. To see that (2') holds, consider the bases (maximal independent sets) $\mathcal{B} \subseteq \mathcal{I}$. By definition, each basis $B \in \mathcal{B}$ is a maximal forest over $G$. Since a spanning tree of a graph with $n$ nodes must needs have $n-1$ edges, we have $|B| = |V| - k$, where $k$ is the number of components of $G$. This is the same for every $B \in \mathcal{B}$, which proves property (2'). Any matroid given by a graph $G$, denoted by $\mathfrak{M}(G)$, is called a \textit{graphic matroid}.

\subsubsection{Random graphs}
Since generating random graphic matroids will require us to generate random graphs, let us take a look at some of the options available to us for this. Luckily for us, random graphs has been an area of extensive study for more than sixty years, and several models with different properties exist.

The Erdős-Rényi (ER) model (also known as Erdős-Rényi-Gilbert~\cite{fienberg-2012}) picks uniformly at random a graph from among the $\binom{\binom{n}{2}}{M}$ possible graphs with $n$ nodes and $M$ edges, or, alternatively, constructs a graph with $n$ nodes where each edge is present with some probability $p$~\cite{erdos-1959, gilbert-1959}. This model produces mostly disconnected graphs, and the size distribution of its components with respect to the number of edges has been studied extensively. With $n$ nodes and fewer than $\frac{n}{2}$ edges, the resulting graph will almost always consist of components that are small trees or contain at most one cycle. As the number of edges exceeds $\frac{n}{2}$, however, the so-called ``giant'' component of size $\mathcal{O}(n)$ emerges, and starts to absorb the smaller components~\cite{janson1993birth}. The ER model is the oldest and most basic random graph model, and is often referred to simply as the random graph, denoted by $G(n,p)$.

Variations of the ER model have been developed by physicists and network scientists to produce phenomena commonly seen in real-world networks~\cite{fienberg-2012}. These variations include the Barabási-Albert model, which grows an initial connected graph using preferential attachment (a mechanism colloquially known as ``the rich get richer''), in which more connected nodes are more likely to receive new connections. This results in graphs in which a small number of nodes (``hubs'') have a significantly higher degree than the rest, creating a power-law distribution of node degrees. This property is known as scale-freeness and is thought to be a characteristic of the Internet~\cite{barabasi-albert}. 

Another approach is the Watts-Strogatz model, which starts with a ring lattice, a regular graph with $n$ nodes and $k$ edges per node, and then rewires each edge with some probability $p$. By changing $p$, one is able to `tune' the graph between regularity (p=0) and disorder (p=1). For intermediate values of $p$, Watts-Strogatz produces so-called ``small-world'' graphs, which exhibit both a high degree of clustering (how likely two nodes with a common neighbor are to be adjacent), and short average distance between nodes. This phenomenon is found in many real-world networks, such as social systems or power grids~\cite{Watts-1998}.

\subsubsection{Random graphic matroids}
We will use the Graphs.jl library~\cite{Graphs2021} for handling graphs in Matroids.jl. 